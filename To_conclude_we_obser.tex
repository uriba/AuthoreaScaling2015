To conclude, we observe that $\approx 9\%$ of the change in the proteome composition across conditions is the result of scaling with growth rate of many proteins.
Further discussion of the fraction of variability explained can be found in \ref{corrthreshold}.
This scaling with growth rate is coordinated, meaning the proteins maintain their relative abundances.
Finally, this fraction of proteins encompasses many proteins from different functional groups and cellular mechanisms.


\subsection{Theoretical model}
What is the simplest way to explain the observed coordinated growth rate dependence, that is, can this behavior be explained without invoking parameter tuning or complex layers of regulation?
In an attempt to parsimoniously explain this wide-spread, coordinated, positive response, we have constructed a minimalistic model that is able to reproduce these observations as the outcome of redistribution of resources of the bio-synthesis machinery.
Before presenting the model mathematically, we give a brief intuitive depiction.


The model assumes that, under favorable growth conditions, the cell actively down-regulates some proteins that are needed in harsher conditions, as is illustrated in Figure \ref{fig:model}.
As a result, the fraction of each of the rest of the proteins out of the proteome increases, showing the same relative ratios, as long as there is no specific regulation.
The growth rate thus increases, as the ratio of bio-synthetic machinery to the rest of the proteome increases, as is depicted in Figure \ref{fig:model}B.
To demonstrate the idea concretely, one could think about the down regulation of the lac operon in the presence of Glucose, that alleviates the need to transcribe and translate lactose metabolism genes and coincides with faster growth.
