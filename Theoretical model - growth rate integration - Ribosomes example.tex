To illustrate this assumption concretely, one could think about the synthesis of polypeptides.
If a cell has $R$ actively translating ribosomes, each of which synthesizing polypeptides at a rate of $\eta \approx 20$ amino acids per second, the bio-synthetic capacity of the cell will be limited to $\approx \eta R$ amino acids per second.
If the total amount of protein in that same cell is $P$ (measured in amino acids count), it follows that the time it will take the actively translating ribosomes to synthesize the proteins for an identical daughter cell is $\tau \approx \frac{P}{\eta R}$, up to a $\ln(2)$ factor resulting from the fact that the ribosomes also synthesize more ribosomes during the replication process and that these new ribosomes will increase the total rate of polypeptides synthesis.

  