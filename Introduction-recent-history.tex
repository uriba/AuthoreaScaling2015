In the last two decades, with the development of the ability to measure genome-wide expression levels, it was found that changes in gene expression (measured through mRNA levels and promoter-reporter libraries) as a function of growth rate is not limited to ribosomes and ribosomal genes.
In \emph{E.coli}, the expression of catabolic and anabolic genes is coordinated with growth rate, and suggested to be mediated by cAMP \cite{Saldanha2004}.
In \emph{S.cerevisiae}, it was shown that a surprisingly large fraction of the genome changes its expression levels in response to environmental conditions in a manner strongly correlated with growth rate \cite{Keren2013a,Gasch2000,Castrillo2007,Gerosa2013}.
Studies examining the interplay between global and specific modes of regulation, suggested that global factors play a major role in determining the expression levels of genes \cite{Gasch2000,Klumpp2009a,Scott2010,Berthoumieux2013,Keren2013a,Gerosa2013}.
In \emph{E.coli}, this was mechanistically attributed to changes in the pool of RNA polymerase core and sigma factors \cite{Klumpp2008}.
In \emph{S.cerevisiae}, it was suggested that differences in histone modifications around the replication origins \cite{Regenberg2006} or translation rates \cite{Gasch2000} across conditions may underlie the same phenomenon.
Important advancements in \emph{E.coli} were achieved by analyzing measurements of fluorescent reporters through a simplified model of gene expression built upon the empirical scaling with growth rate of different cell parameters (such as gene dosage, transcription rate and cell size)\cite{Klumpp2009a}.
These studies suggest that the expression of all genes changes with growth rate, with different factors and architectures of regulatory networks yielding differences in the direction and magnitude of these changes. 