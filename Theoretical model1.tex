\subsection{Simple considerations predict passively driven increase in the concentration of proteins as a function of the growth rate}
What is the simplest way to model the differences in the proteome composition of two populations of cells, one growing in a permissive environment, and the other facing a more challenging growth condition?
In an attempt to parsimoniously analyze such differences, we have constructed a minimalistic model that is able to predict the behavior of non-differentially regulated genes across different growth conditions as an outcome of the redistribution of resources of the bio-synthesis machinery.
Before presenting the model mathematically, we give a brief intuitive depiction.

The model assumes that, under favorable growth conditions, the cell actively down-regulates some proteins that are needed in harsher conditions, as is illustrated in Figure \ref{fig:model}.
As a result, the fraction of each of the rest of the proteins out of the proteome is high (compared with the harsh condition), showing the same relative ratios, as long as there is no specific regulation.
The growth rate is thus also high, compared with the harsh condition, as the ratio of bio-synthetic machinery to the rest of the proteome is higher, as is depicted in Figure \ref{fig:model}B.
The growth rate thus reflects the amount of extra bio-synthesis work a cell needs to perform in order to synthesize essential proteins needed under its growth environment compared with an ideal growth condition.
To demonstrate the idea concretely, one could think about the down regulation of the lac operon in the presence of Glucose, that alleviates the need to transcribe and translate lactose metabolism genes and coincides with faster growth.