\subsection{Simple considerations predict passively driven increase in the concentration of proteins as a function of the growth rate}
What is the simplest way to model the differences in the proteome composition of two populations of cells, one growing in a permissive environment, and the other facing a more challenging growth condition?
In an attempt to parsimoniously analyze such differences, we have constructed a minimalistic model that predicts the behavior of non-differentially regulated genes across different growth conditions. The changes in levels are an outcome of the redistribution of resources of the bio-synthesis machinery.
Before presenting the model mathematically, we give a brief intuitive depiction.

The model assumes that, under favorable growth conditions, the cell actively down-regulates some proteins that were needed in harsher conditions but not needed in the favorable condition, as illustrated in Figure \ref{fig:model}.
As a result, the fraction of each of the rest of the proteins out of the proteome is increased compared with the harsh condition, as long as there is no gene-specific regulation. All those proteins increase their levels but are expected to show the same relative ratios among each other after the increase as they were before. 
The growth rate is also expected to increase in comparison with the harsh condition, as the ratio of bio-synthetic machinery to the rest of the proteome is higher, as  depicted in Figure \ref{fig:model}B.
The growth rate is dependent on the amount of bio-synthesis a cell needs to perform in order to synthesize the proteins needed under its growth environment. 
To demonstrate the idea concretely, one could think about the down regulation of the lac operon in the presence of Glucose. This situation alleviates the need to transcribe and translate lactose metabolism genes and leads to faster growth.