\subsection{Simple considerations predict passively driven increase in the fraction of proteins as a function of the growth rate}
What is the simplest way to model the differences in the proteome composition of two populations of cells, one growing in a permissive environment, and the other facing a more challenging growth condition?
In an attempt to parsimoniously analyze such differences, we have constructed a minimalistic model that predicts the behavior of non-differentially regulated genes across different growth conditions.
Before presenting the model mathematically, we give a brief intuitive depiction.

The model assumes that, under a favorable growth condition, the cell actively down-regulates some proteins that are only needed in harsher conditions, as illustrated in Figure \ref{fig:model}.
The down regulation of the lac operon in the presence of Glucose is a prominent example for this phenomenon. 
As a result, the fraction of all other proteins out of the proteome is increased compared to the harsh condition.
All those proteins increase their levels but are expected to show the same relative ratios between each other in all conditions. 
Specifically, the proteins forming the bio-synthetic machinery increase their levels and therefore their fraction out of the proteome.
Thus, the ratio of the bio-synthetic machinery to the proteome increases.
The growth rate is dependent on the amount of bio-synthesis a cell needs to perform in order to synthesize the proteins needed under its growth environment.
The increase in ratio of bio-synthetic machinery to proteome thus results in an expected increase in the growth rate, as  depicted in Figure \ref{fig:model}B.
In our example of the lac operon in the presence of glucose, the down regulation of lactose metabolism genes leads to faster growth as more bio-synthetic genes can be expressed instead.