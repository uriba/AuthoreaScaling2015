\subsection{Simple considerations predict passively driven increase in the concentration of proteins as a function of the growth rate}
What is the simplest way to model the differences in the proteome composition of two populations of cells, one growing in a permissive environment, and the other facing a more challenging growth condition?
In an attempt to parsimoniously analyze such differences, we have constructed a minimalistic model that predicts the behavior of non-differentially regulated genes across different growth conditions.
Before presenting the model mathematically, we give a brief intuitive depiction.

The model assumes that, under favorable growth conditions, the cell actively down-regulates some proteins that were needed in harsher conditions but not needed in the favorable condition, as illustrated in Figure \ref{fig:model}.
For example, one could think about the down regulation of the lac operon in the presence of Glucose. 
As a result, the fraction of each of the rest of the proteins out of the proteome is increased compared with the harsh condition, as long as there is no gene-specific regulation.
All those proteins increase their levels but are expected to show the same relative ratios among each other after the increase as they were before. 
Because the growth rate is dependent on the amount of bio-synthesis a cell needs to perform in order to synthesize the proteins needed under its growth environment, and as, under the favorable condition, the down regulation of the expression of proteins needed for growth in harsher conditions results in an increase in the ratio of bio-synthetic machinery to the rest of the proteome, the growth rate is expected to increase, as  depicted in Figure \ref{fig:model}B.
The increase is growth rate is consistent with our example of the down regulation of the lac operon in the presence of glucose, down regulation that alleviates the need to transcribe and translate lactose metabolism genes and leads to faster growth.