Early on it was found that the expression of some genes is coordinated with growth rate, rather than with the specific environment.
Classic experiments in bacteria, by researchers from what became known as the Copenhagen school, have shown that ribosome concentration (inferred from the RNA to protein ratio in cells) increases in proportion to growth rate \cite{Schaechter1958}.
The observed increase in concentration has been interpreted to indicate that, given that translation rates and the fraction of active ribosomes remain relatively constant across conditions, a larger fraction of ribosomes out of the proteome is needed in order to achieve faster growth \cite{Neidhardt1999,Dennis2004,Zaslaver2009}. The search for mechanisms in \emph{E.coli} that underlie this observation yielded several candidates.
Specifically, coordination between ribosome production and growth rate was attributed both to the pools of ppGpp and iNTP \cite{Murray_2003,Bosdriesz_2015}, and the tRNA pools through the stringent response \cite{Chatterji2001,Brauer2008a}.
For a more thorough review see \cite{Nomura1984}.