\subsubsection{Calculation of protein concentration}
\label{protconc} 

In this study, we use the mass ratio of a specific protein to the mass of the entire proteome, per cell, as our basic measure for the bio-synthetic resources a specific protein consumes out of the bio-synthetic capacity of the cell.
We find this measure to be the best representation of the meaning of a fraction a protein occupies out of the proteome.
However, we note that if initiation rates are limiting (e.g. if RNA polymerase rather than ribosomes become limiting), and not elongation rates, then using molecule counts ratios (the number of molecules of a specific protein divided by the total number of protein molecules in a cell) rather than mass ratios may be a better metric.
We compared these two metrics and, while they present some differences in the analysis, they do not qualitatively alter the observed results.

There are different, alternative ways to assess the resources consumed by a specific protein out of the resources available in the cell.
On top of the measures listed above, one could consider either the total mass or molecule count of a specific protein out of the biomass, rather than the proteome, or out of the dry weight of the cell, both of which vary with the ratio of total protein to biomass or dry weight which was neglected in our analysis.
Moreover, one can consider specific protein mass or molecule count per cell, thus reflecting changes in cell size across conditions.
Our analysis focused on the relations between different proteins and resource distribution inside the proteome, and thus avoided such metrics.