\section{Discussion}
We characterized a significant, coordinated response in \emph{E.coli} between many proteins and the growth rate.
This response spans proteins from various functional groups and is not related to the specific medium of growth.
A similar phenomena is observed for \emph{S.cerevisiae} as was reported in \cite{Keren2013a} and may thus be conserved across various organisms and domains of life.
Our analysis suggests that, while changes in the proteome composition may seem complex, they can, to some extent, be attributed to the linear increase with growth rate of a large number of proteins, at the expense of a few, down-regulated proteins.
The well studied scaling of ribosomes concentration with growth rate can be considered one manifestation of the more general phenomena we describe here.
We find that this response, when considering ribosomal proteins fraction out of the proteome, is not unique and is, in fact, shared with many other proteins spanning different functional groups.

We re-introduce the notion of intrinsic affinity for expression, first presented in \cite{Maaloe1969}, though not widely known.
We show that integrating this notion with the limited bio-synthesis capacity of cells results in a parsimonious mechanism that can explain the response observed in recent experimental data, data that was unavailable at the time this notion was first introduced, nearly 50 years ago.

The framework we present emphasizes the importance of accounting for global factors, that are reflected in the growth rate, when analyzing gene expression and proteomics data.
Specifically, we suggest that the default response of a protein (that is, the change in the observed expression of a protein, given that no specific regulation was applied to it) is to linearly increase with growth rate.
We note that the exact parameters of this dependency may depend on factors such as the degradation rate and the global rate of bio-synthesis mechanisms, as well as the specific affinity of the protein.
We point out that, as non-differentially regulated proteins maintain their relative abundances, one can overcome the lack of knowledge of these factors and use the scaling of most of the proteins in the proteome to infer these expected default dependency parameters as is demonstrated in Figure \ref{fig:randpred}.

Interestingly, our model demonstrates that no specific control mechanisms need to exist in order to achieve a linear correlation between ribosomal proteins and the growth rate.
However, many such mechanisms have been described before \cite{Nomura1984}.
We stress that the model does not contradict the existence of such mechanisms.
Ribosomal proteins expression control mechanisms may still be needed to achieve faster response under changing environmental conditions or a tighter regulation to avoid unnecessary production and reduce translational noise.
Furthermore, such mechanisms may be crucial for synchronizing the amount of rRNA with ribosomal proteins as the two go through different bio-synthesis pathways.
Nevertheless, the fact that many non-ribosomal proteins share the same response as ribosomal proteins do, poses interesting questions regarding the scope of such control mechanisms, their necessity and the trade-offs involved in their deployment.