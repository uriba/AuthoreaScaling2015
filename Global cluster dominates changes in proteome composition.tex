\subsubsection{Changes in the proteome across environmental conditions are dominated by proteins that are positively correlated with growth rate}
Lastly, we assessed the significance of the positive correlation of proteins with growth rate, out of the total change in proteome composition across conditions.
To that end, we summed the concentrations of all of the proteins that are strongly correlated with growth rate across the conditions measured and plotted their total concentration against the growth rate in Figure \ref{fig:globalgrcorr}.
Both data sets show that the concentration of these proteins change $\approx 2$ fold across an $\approx 5$ fold change in the growth rate under the different growth conditions.
Moreover, most of the variability of the total concentration of these proteins can be explained by the growth rate ($R^2$ of $0.8$ in H and $1$ in V). 
For further analysis of the differences between the two data sets see section \ref{heinemannchemo}.
Importantly, the strongly correlated proteins form a large fraction of the proteome, exceeding $50\%$ of the proteome, mass-wise, at the higher growth rates measured.
Thus, when considering the changes in proteome composition across conditions, we find that, at higher growth rates, more than $50\%$ of the proteome composition is affected by the coordinated response of the same group of proteins with growth rate.

However, despite the magnitude of this phenomena, when calculating the fraction of the total variability in the proteome that is accounted for by this linear response, we observe that only $\approx 9\%$ of the change in the proteome composition across conditions results from linear scaling with growth rate of the proteins that share a coordinated, positive response with the growth rate.
A lot of this seeming difference results from the fact that a single linear response captures only a fraction of the variability of these proteins across the different growth conditions, possibly due to measurement noise.
Further discussion of the fraction of variability explained can be found in \ref{corrthreshold}.
The noise in current whole proteome measurement techniques make it difficult to distinguish between proteins that scale coordinately, as is predicted by our model, and proteins that scale differentially, but within measurement uncertainty.
Thus, it is unclear to what extent the effect we predict affects actual protein concentrations versus their possible individual up regulation with growth rate.
We expect future improvements in the accuracy of whole proteome measurements to quantitatively reveal the importance of passive coordinated scaling with growth rate in shaping the proteome composition. These coming improvements in accuracy will enable better testing of the scope and validity of the model presented here.