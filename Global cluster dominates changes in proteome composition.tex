\subsubsection{Changes in the proteome across environmental conditions are dominated by proteins that are positively correlated with growth rate}
To assess the significance of the positive correlation of proteins with growth rate, out of the total change in proteome composition across conditions, we summed the concentrations of all of the proteins that are strongly correlated with growth rate across the conditions measured and plotted their total concentration against the growth rate in Figure \ref{fig:globalgrcorr}.
Both data sets show that the concentration of these proteins change $\approx 2$ fold across a $\approx 5$ fold change in the growth rate under the different growth conditions.
This change is smaller than the change predicted by our basic model and the deviations may result from the effects of degradation and varying biosyntheis rates, as is discussed in sections \ref{degradation} and \ref{slowbiosynthesis}.
Moreover, most of the variability of the total concentration of these proteins can be explained by the growth rate ($R^2$ of $\hGlobalSumRsq$ in the data set from \cite{Heinemann2015} and $\vGlobalSumRsq$ in the data set from \cite{Peebo_2015}). 
For further analysis of the differences between the two data sets see section \ref{heinemannchemo}.
Importantly, the strongly correlated proteins form a large fraction of the proteome, exceeding $50\%$ of the proteome, mass-wise, at the higher growth rates measured.
This is a much higher fraction than the one obtained for randomized data sets ($\lt 4\%$, as is further discussed in section \ref{randanalysis})
Thus, when considering the changes in proteome composition across conditions, we find that, at higher growth rates, more than $50\%$ of the proteome composition is affected by the coordinated response of the same group of proteins with growth rate.

Despite the magnitude of this phenomena, the fraction of the total variability in the proteome that is accounted for by this linear response is only $\approx 9\%$  in the data set from \cite{Heinemann2015} and even lower in \cite{Peebo_2015} as can be seen in Figure \ref{fig:threshold}.
While this fraction is low, it is still much higher than the equivalent $2\%$ obtained for a randomized data set based on the data from \cite{Heinemann2015}, as is described in section \ref{randanalysis}.
This relatively low explained variability fraction is primarily the result of two factors: the linear response applies only to $\approx0.4$ of the proteins, leaving the rest of the proteins with no prediction, and experimental noise in whole proteome measurement techniques, estimated at $\approx25\%$.
Further discussion of the fraction of variability explained can be found in \ref{corrthreshold}.