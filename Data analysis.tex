\subsubsection{Changes in the proteome across environmental conditions are dominated by proteins that are positively correlated with growth rate}
Lastly, we assessed the significance of the positive correlation with growth rate of proteins, out of the total change in proteome composition across conditions.
To that end, we summed the concentrations of all of the proteins that are strongly correlated with growth rate across the conditions measured and plotted their total concentration against the growth rate in Figure \ref{fig:globalgrcorr}.
Both data sets show that the concentration of these proteins change significantly across the different conditions ($\approx 2$ fold change in total concentration across $\approx 5$ fold change of the growth rate).
Moreover, most of the variability of the total concentration of these proteins can be explained by the growth rate ($R^2$ of $0.79$ in H and $0.99$ in V). 
For further analysis of the differences between the two data sets see section \ref{heinemannchemo}.
Importantly, the strongly correlated proteins form a large fraction of the proteome mass-wise, exceeding $50\%$ of the proteome at the higher growth rates measured.
Thus, when considering the changes in proteome composition across conditions, we find that, at higher growth rates, more than $50\%$ of the proteome composition is affected by the positive correlation with growth rate of the same group of proteins.
\begin{comment}
\emph{ToDo: This analysis, unlike the one done in Leeat's work, does not try to identify the scaling between every two conditions and then cluster genes according to conditions where specific regulation occurred, modifying the analysis in that direction can potentially yield stronger results as, instead of identifying only the proteins for which no regulation takes place across ALL conditions, it can identify all level-changes reflecting no regulation out of all the level-changes of all the proteins across all conditions}
\end{comment}