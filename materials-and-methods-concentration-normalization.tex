\subsubsection{Normalizing protein concentration across conditions}
\label{concacrossconds} 

When comparing the response of different proteins to the growth rate, a normalization is required to compensate for differences in their mean concentrations.
For example, consider two proteins, $A$ and $B$ measured under two conditions, $C_1$ and $C_2$.
Assume that the measured fractions out of the proteome of these two proteins under the two conditions were $0.001$ and $0.002$ for $A$ under $C_1$ and $C_2$ respectively, and $0.01$ and $0.02$ for $B$ under $C_1$ and $C_2$ respectively.
These two proteins therefore respond in the same way across the two conditions, namely, they double their fraction in the proteome in $C_2$ compared with $C_1$.
The normalization procedure scales the data so as to reveal this identity in response.
Dividing the fraction of each protein out of the proteome by the average fraction of that protein across conditions, we get normalized fractions that are $\frac{2}{3}$ for both $A$ and $B$ at $C_1$ and $\frac{4}{3}$ for both $A$ and $B$ at $C_2$ showing their identical response.

This normalization procedure has been used prior to calculating the slopes of the regression lines best describing the change in fraction out of the proteome of every protein as a function of the growth rate.
Furthermore, when analyzing the variability explained by linear regression on the sum of concentrations of all proteins presenting a high correlation with the growth rate, the same normalization procedure was made in order to avoid biases resulting from the high abundance of a few proteins in that group.