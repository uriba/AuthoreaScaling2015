\subsubsection{Normalizing protein concentrations across conditions}
\label{concacrossconds} 
Our analysis aims at identifying proteins that share similar expression patterns across the different growth conditions.
For example, consider two proteins, $A$ and $B$ measured under two conditions, $c_1$ and $c_2$.
Assume that the measured fractions out of the proteome of these two proteins under the two conditions were $0.001$ and $0.002$ for $A$ under $c_1$ and $c_2$ respectively, and $0.01$ and $0.02$ for $B$ under $c_1$ and $c_2$ respectively.
These two proteins therefore share identical responses across the two conditions, namely, they double their fraction in the proteome in $c_2$ compared with $c_1$.

The normalization procedure scales the data so as to reveal this identity in response.
Dividing the fraction of each protein out of the proteome by the average fraction of that protein across conditions yields the normalized response.
It the example, the average concentration of $A$ across the different conditions is $0.0015$ and the average concentration of $B$ is $0.015$.
Thus, dividing the concentration of every protein by the average concentration across conditions of that same protein yields:
\[
A'_{c_1}=\frac{A_{c_1}}{\bar{A}}=\frac{0.001}{0.0015}=\frac{2}{3}=\frac{0.01}{0.015}=\frac{B_{c_1}}{\bar{B}}=B'_{c_1}
\]
for $c_1$ and:
\[
A'_{c_2}=\frac{A_{c_2}}{\bar{A}}=\frac{0.002}{0.0015}=\frac{4}{3}=\frac{0.02}{0.015}=\frac{B_{c_2}}{\bar{B}}=B'_{c_2}
\]
for $c_2$ showing $A$ and $B$ share identical responses across $c_1$ and $c_2$.

The general normaliztion procedure thus divides the concentration of protein $i$ under condition $c$, $p_i(c)$ by the average concentration of protein $i$ across all of the conditions in the data set, $\bar{p}_i$, to give the normalized concentration under condtion $c$, $p'_i(c)=\frac{p_i(c)}{\bar{p}_i}$.

This normalization procedure has been applied prior to calculating the slopes of the regression lines best describing the change in fraction out of the proteome of every protein as a function of the growth rate.
Furthermore, when analyzing the variability explained by linear regression on the sum of concentrations of all proteins presenting a high correlation with the growth rate, the same normalization procedure was made in order to avoid domination by the high abundance of a few proteins in that group.