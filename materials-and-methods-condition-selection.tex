\subsubsection{Filtering out conditions from the Heinemann data set}
\label{heinemanncond} 

The \cite{Heinemann2015} data set contains proteomic data measurements under 22 different environmental conditions.
However, our model assumes exponential growth, implying that measurements taken at stationary phase are expected to differ from simple extrapolation of the model to zero growth rate.
Therefore, the two measurements of stationary phase proteomics were excluded from our analysis.

Out of the conditions measured in the \cite{Heinemann2015} data set, two conditions included amino acids in the media and presented much faster growth rate than the rest of the conditions (growth in LB media and in glycerol supplemented with AA, with growth rates of $1.9[h^{-1}]$ and $1.27[h^{-1}]$ respectively, compared with a range of $0.12-0.66$ for the other conditions).
This asymmetry in the distribution of growth rates caused inclusion of these conditions to dominate the analysis due to its effect on the skewness of the distribution of growth rates ($\gamma_1=-0.5$ for the growth rates excluding LB and AA supplemented glycerol vs. $\gamma_1=2.3$ with LB and AA supplemented glycerol) reducing the statistical power of the other conditions.
While including the data on growth in these conditions does not qualitatively change the observed results, such analysis is much less statistically robust.
We have therefore omitted growth in LB and in AA supplemented glycerol in the main analysis.
We present the analysis including these conditions in section \ref{lbanalysis}.

