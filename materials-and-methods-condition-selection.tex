\subsubsection{Filtering out conditions from the Heinemann data set}
\label{heinemanncond} 

The \cite{Heinemann2014} data set contains proteomic data measurements under 19 different environmental conditions.
However, some of these conditions are expected to affect the proteome composition in ways not related to the actual growth rate.
Specifically, our model assumes that bio-synthesis rates are condition independent and, at most, change only with growth rate.
That will likely not be the case under specific kinds of conditions such as osmotic pressure, low pH and heat stress.
Additionally, our model assumes exponential growth, implying that measurements taken at stationary phase are expected to differ from simple extrapolation of the model to zero growth rate.
We have therefore omitted such conditions from our analysis and analyzed only those conditions under which our assumptions for bio-synthesis rates being either constant or have a simple relation to the growth rate do hold \todo{need to add these analyses to the SI}.

As, out of the conditions measured in the \cite{Heinemann2014} data set, growth in LB media presented a much faster growth rate than the rest of the conditions measured, it dominated the behavior and trends calculated, when included in the analysis.
While including the data on growth in LB does not qualitatively changes the observed results, such analysis is less statistically robust.
We have therefore omitted LB growth data in the main analysis yet present it in the SI.
Including LB growth results in a much smaller set of proteins with a strong positive correlation with growth (as many of the proteins in that group in the slower conditions get down-regulated in LB, significantly reducing their Pearson correlation with growth rate).
On the other hand, the proteins that remain strongly positively correlated with growth rate when LB is included in the analysis show a higher correlation compared with the strongly correlated with growth rate proteins under the slower conditions.
Furthermore, despite the decrease in the number of proteins that are strongly positively correlated with growth when LB is included in the analysis, these proteins form up to $50\%$ of the proteome under LB.