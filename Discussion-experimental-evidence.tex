We analyze two recent whole proteome data sets to explore the scope and validiy of our model.
We characterize a coordinated response in \emph{E.coli} between many proteins and the specific growth rate.
This response spans proteins from various functional groups and is not related to the specific medium of growth.
A similar phenomena is observed for \emph{S.cerevisiae} as was reported in \cite{Keren2013a} and may thus be conserved across various organisms and domains of life.
Our analysis suggests that, while changes in the proteome composition may seem complex, for a large number of proteins and under many conditions, they can be attributed to a linear, coordinated, increase with growth rate, at the expense of other, down-regulated proteins.
The well studied scaling of ribosomes concentration with growth rate can be considered one manifestation of the more general phenomena we describe here.
We find that this response is not unique to ribosomal proteins but is, in fact, shared with many other proteins spanning different functional groups.
Furthermore, the linear dependence slope and explained variability of fraction levels of proteins explained by linear correlation with growth rate is similar among the ribosomal proteins versus all the proteins with high correlation with the growth rate.