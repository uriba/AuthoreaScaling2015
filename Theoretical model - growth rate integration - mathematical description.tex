Formally, we define the group of bio-synthesis genes, $G_B$, such that, for every gene that belongs to this group, $k \in G_B$, its affinity, $w_k(c)$ is constant regardless of the condition, $c$.
\begin{equation}
  \label{eq:biosynth-def}
  w_k(c)=w_k
\end{equation}

To keep our notations short, we will define the condition independent sum over all of these bio-synthesis genes as the constant:
\[
W_B = \sum_{k\in G_B}w_k
\]

The doubling time under a given condition, $\tau(c)$, will be proportional to the ratio of total protein to bio-synthesis protein under that condition, with the proportionality constant $T_B$:
\begin{equation}
  \label{eq:gr-ratio}
  \tau(c) = T_B\frac{P(c)}{\sum_{k\in G_B}P_k(c)}=T_B\frac{\sum_jw_j(c)}{W_B}
\end{equation}
Therefore, the model reproduces, for conditions requiring larger amounts of non-bio-synthetic proteins (i.e. higher values in the sum over $w_j$), an increase in the doubling time.