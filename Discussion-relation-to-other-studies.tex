\subsection{Relation to other studies}
The findings in this study support and broaden the findings in other recent studies.
Specifically, for \emph{S.cerevisiae} a few recent studies found that the concentration of the majority of the proteins is coordinated across conditions \cite{Keren2013a,Gasch2000,Brauer2008a} and increases with growth rate.
In principle, the model we suggest here can be applied to any microorganism and may thus also serve as a potential explanation for the phenomena observed in these studies.


Other recently published studies in \emph{E.coli} have suggested different models and in some cases have results  and predictions that do not coincide with those presented in this study.
Notably, in \cite{Klumpp2009a} the opposite behavior for unregulated genes is predicted.
A few differences can explain this seeming discrepancy, among which are different ranges of growth rates observed (the growth rates in this study lie in a range smaller than the growth rates in \cite{Klumpp2009a}), different strains used and very different methods for deducing the expected behavior (ref SI for further discussion?).


Many studies monitored the ribosome concentration in cells and its interdependence with growth rate \cite{Scott2010,Bremer1987,Schaechter1958,1974,Zaslaver2009,Bremer1987}.
While in all of these studies a linear dependence of ribosome concentration with growth rate was observed, in some cases different response strengths were found, compared with the observations in this study.
A discussion for various reasons that may underlie these differences is in \ref{ribosomeconc}.
Conducting similar analysis on the data sets used in this study reveals that, while a linear relation exists, it is not unique to ribosomal proteins but is in fact shared among many more genes.
Furthermore, the variability in the concentration levels of proteins explained by linear correlation with growth rate is similar among the ribosomal proteins versus all the proteins with high correlation with the growth rate as is shown in Figure \ref{fig:ribsnonribs}.

The expected availability of increasing amounts of whole proteome data sets, with higher accuracy levels, following advances in experimental methods and measurement techniques, will enable further investigation of the details of cellular resource distribution and the responses microorganisms have for various environmental changes.
The analysis of such future data sets will shed more light on the relative roles of carefully tuned response mechanisms versus global, passive effects in shaping the proteome composition under different growth environments.