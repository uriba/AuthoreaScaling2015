\subsubsection{Protein degradation differentiates between measured growth rate and biomass synthesis rate}
\label{degradation}
In the following two sections we analyze the effects of expanding our model to account for two biological effects: protein degradation and changes in the rates at which molecular machines operate.

The model we developed predicts that when the growth rate approaches zero, the fraction of every protein with constant affinity also approaches zero.
This approach to zero applies specifically to the biosynthesis genes, that have constant affinities according to our assumptions.
However, it is known that the fractions of these proteins, and specifically of ribosomal proteins does not drop to zero when the growth rate approaches zero \cite{ingraham1983growth,Pedersen1978a}.
We can account for this phenomenon by including protein degradation in our model.

We assume the degradation rate to be constant for all genes and conditions.
The \emph{observed} growth rate, $g$, is then the amount of proteins produced \emph{minus} the amount of proteins degraded.
To illustrate, at zero growth rate, the implication is not that no proteins are produced, but rather that proteins are produced at exactly the same rate as they are degraded.

Integrating this notion into the model means that the bio-synthesis capacity needs to suffice to re-synthesize all the degraded proteins.
Hence, where the equations previously referred to the cellular growth rate, $g$, as the indicator of protein synthesis rate, they should in fact refer to the cellular growth rate plus the degradation rate, as that is the actual rate of protein synthesis.
If we denote by $\alpha$ the degradation rate, Equation \ref{eq:final-conc} should thus be rewritten as:
\begin{equation}
  \label{eq:final-conc-deg}
  p_i(c)=Aw_i(c)(g(c)+\alpha)
\end{equation}
This equation predicts linear dependence of the fraction of unregulated proteins on the growth rate, with an intercept with the horizontal axis occurring at minus the degradation rate (Figure \ref{fig:theoreticalpred}).
Thus, at zero growth rate, the fraction of non-differentially regulated proteins out of the proteome is positive, equalling $Aw_i(c)\alpha$.