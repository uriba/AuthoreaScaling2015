\subsubsection{Protein degradation differentiates between measured growth rate and biomass synthesis rate}
The model can be expanded to include the expected effects of proteome degradation.
Accounting for protein degradation affects the predicted concentration of non-differentially regulated proteins at zero growth rate.

Assuming that protein degradation acts on all proteins in the same way, and that it is invariant in the growth condition, the effect of protein degradation can be understood as follows: at any time, some fraction of the entire proteome is degraded.
Therefore, the \emph{observed} growth rate, $g$, is, in fact, the amount of proteins produced \emph{minus} the amount of proteins degraded.
To illustrate, if a cell does not grow, the implication is not that no proteins are produced, but rather that proteins are produced at exactly the same rate as they are degraded.

Integrating this notion into the model means that, where the equations previously referred to the cellular growth rate, $g$, as the indicator of protein synthesis rate, they should in fact refer to the cellular growth rate plus the degradation rate, as that is the real rate of protein synthesis.
Therefore, if we denote by $\alpha$ the degradation rate (assuming for now equal degradation rates for all genes and under all conditions), Equation \ref{eq:final-conc} should be rewritten as:
\begin{equation}
  \label{eq:final-conc-deg}
  p_i(c)=Aw_i(c)(g(c)+\alpha)
\end{equation}
This equation predicts linear dependence of the concentration of unregulated proteins with the growth rate, with an intercept with the horizontal axis occuring at minus the degradation rate.
Degradation can thus explain why concentrations of non-differentially regulated proteins do not drop to zero when the growth rate is zero.