\subsubsection{The expression level of a protein can be decomposed into gene specific control and global
expression machinery availability}
The composition of the proteome can in principle be determined by a large number of parameters.
For example, given that an organism expresses 1000 genes across 10 different growth conditions, one could imagine that controlling the expression pattern of all genes across all conditions will require 10,000 parameters.
Our model proposes an underlying architecture that drastically reduces this amount of parameters, implying that cells control most of the composition of their proteome through fewer degrees of freedom than might be naively expected.

The model separately considers the resulting fraction of every protein out of the proteome as the product of two control mechanisms:
\begin{enumerate}
\item Protein/gene specific controls which only affect the individual protein under a given condition.
These include the gene associated promoter affinity, 5'-UTRs, ribosomal binding site sequence, as well as the presence of transcription factors and riboswitches that react with the relevant gene.
We note that while a given transcription factor may affect many genes, the presence or absence of its relevant binding sequence is gene specific, making this control mechanism gene specific in the context of our analysis.
  While some of these controls (such as the ribosomal binding sites) are static, and therefore condition independent, others are dynamic and will differ across different environmental conditions (such as transcription factors state, for genes that are affected by them).
\item Global expression control based on the availability of bio-synthetic resources, including RNA polymerase, co-factors, ribosomes, amino-acids etc.
  All of these factors can potentially differ across different environmental conditions and no gene can avoid the consequences of changes in them.
\end{enumerate}

In the model, every gene is given an 'affinity-for-expression' (or 'intrinsic-affinity') score that encapsulates its tendency to attract the bio-synthetic machinery, as was first suggested in \cite{Maaloe1969}.
This gene-specific value can in principle change across conditions but a key feature is that the gene intrinsic affinity tends to have the same value across many conditions.
Often two values are enough across all conditions, an "off" and "on" value.
We denote the affinity of gene $i$ under growth condition $c$ by $w_i(c)$.
To determine the resulting fraction of every protein, our model assumes that the bio-synthetic resources are distributed among the genes according to those affinities.

Under this framework, assuming each gene has only a finite set of affinities, possibly only one or two (for example, on and off states of the lac operon), the expression pattern under every condition is determined by only setting which, out of the total gene-specific small set of possible affinities, each gene gets under the relevant condition.
Furthermore, given that the selection of expression level for a given gene is driven by some specific environmental cues, one needs only to know what cues are present at each condition in order to fully specify the affinities all genes acquire under that condition, resulting in full characterization of the expected proteome composition.
