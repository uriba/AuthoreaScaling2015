\subsubsection{The concentration of a protein is determined by both gene specific control, and global
expression machinery availability}
The composition of the proteome can potentially be controlled by an enormous amount of parameters.
For example, given that an organism expresses 1000 genes across 10 different growth conditions, one could imagine that controlling the expression pattern of all genes across all conditions will require 10000 parameters.
Our model assumes an underlying architecture that drastically reduces the amount of parameters and thus implies that cells have fewer degrees of freedom in controlling the composition of their proteome.

The model separately considers the resulting concentration of every protein as the product of two control mechanisms:
\begin{enumerate}
\item Protein/gene specific controls such as the gene associated promoter sequence, 5'-UTRs, ribosomal binding site sequence, and factors affecting the specific expression of the gene such as transcription factors and riboswitches that react with the relevant gene.
  While some of these controls (such as, for example, the ribosomal binding sites) are static, and therefore condition independent, others are dynamic and will differ under different environmental conditions (such as transcription factors state).
\item The global availability of bio-synthetic resources in the cell, including availability of RNA polymerase, co-factors, Ribosomes concentration, amino-acids etc.
  All of these factors can potentially differ across different environmental conditions.
\end{enumerate}

For simplicity, the model refers to the fraction of a specific protein out of the proteome, and not to the concentration of that protein in the biomass.
The concentration of a specific protein in the biomass can be calculated given this fraction and the concentration of total protein in the biomass, which is known to be relatively constant \cite{Bremer1987,Scott2014} (for further discussion see \ref{protconc}).

According to the model, every gene, under every environmental condition, is given an 'affinity-for-expression' (or 'intrinsic-strength') score that encapsulates its gene-specific control state under the condition considered, as was first suggested in \cite{Maaloe1969}.
We assume that each gene has only a finite set of affinities, possibly only one or two (for example, the on and off states of the lac operon).
Therefore, the expression pattern under every condition is determined by only setting which, out of the total gene-specific small set of possible affinities, each gene gets under the relevant condition.
It is therefore implied that, given that the selection of expression level for a given gene is driven by some specific environmental cues, one needs only to know what cues are present at each condition in order to fully specify the affinities all genes acquire under that condition, resulting in full characterization of the expected proteome composition.

We denote the affinity of gene $i$ under growth condition $c$ by $w_i(c)$.
Our model assumes that the bio-synthetic resources of the cell (Ribosomes, RNA polymerases, etc.) are distributed among the genes according to their affinities under the condition at hand.