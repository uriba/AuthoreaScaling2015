The notion of affinities can reduce the number of parameters needed to predict expression levels markedly.
For example, given that an organism expresses 1000 genes across 10 different growth conditions, one could imagine that characterizing the expression pattern of all genes across all conditions will require 10000 parameters, (the expression level of every gene under every condition), each of which can potentially vary continuously across some predefined range.
According to our model, each gene has only a finite set of affinities, possibly only one or two (for example, the on and off states of the lac operon).
Therefore, the expression pattern under every condition can be characterized by only specifying which, out of the total gene-specific small set of possible affinities, each gene acquires under every condition.
Moreover, given that the selection of expression level for a given gene is driven by some specific environmental cues, one needs only to know what cues are present at each condition in order to fully specify the affinities all genes acquire under that condition, and thus predict the resulting proteome composition.
