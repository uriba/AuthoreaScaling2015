\subsection{Theoretical model}
What is the simplest way to explain the observed coordinated growth rate dependence, that is, can this behavior be explained without invoking parameter tuning or complex layers of regulation?
In an attempt to parsimoniously explain this wide-spread, coordinated, positive response, we have constructed a minimalistic model that is able to reproduce these observations as the outcome of redistribution of resources of the bio-synthesis machinery.
Before presenting the model mathematically, we give a brief intuitive depiction.


The model assumes that, under favorable growth conditions, the cell actively down-regulates some proteins that are needed in harsher conditions, as is illustrated in Figure \ref{fig:model}.
As a result, the fraction of each of the rest of the proteins out of the proteome increases, showing the same relative ratios, as long as there is no specific regulation.
The growth rate thus increases, as the ratio of bio-synthetic machinery to the rest of the proteome increases, as is depicted in Figure \ref{fig:model}B.
To demonstrate the idea concretely, one could think about the down regulation of the lac operon in the presence of Glucose, that alleviates the need to transcribe and translate lactose metabolism genes and coincides with faster growth.

\subsubsection{The concentration of a protein is determined by both gene specific control, and global expression machinery availability}
For every protein, the model separately considers the resulting concentration as the product of two control mechanisms:
\begin{enumerate}
\item Protein/gene specific controls such as the gene associated promoter sequence, 5'-UTRs, ribosomal binding site sequence, and factors affecting the specific expression of the gene such as transcription factors and riboswitches that react with the relevant gene.
  While some of these controls (such as, for example, the ribosomal binding sites) are static, and therefore condition independent, others are dynamic and may differ under different environmental conditions (such as transcription factors state).
\item The global availability of bio-synthetic resources in the cell, including availability of RNA Polymerase, co-factors, Ribosomes concentration, amino-acids etc.
  All of these factors can potentially differ across different environmental conditions.
\end{enumerate}
For simplicity, the model refers to the fraction of a specific protein out of the proteome, and not to the concentration of that protein in the biomass.
The concentration of a specific protein in the biomass can be calculated given this fraction and the concentration of total protein in the biomass, which is known to be relatively constant \cite{eco-sal,Scott2014} (for further discussion see \ref{protconc}).


According to the model, every gene, under every environmental condition, is given an 'affinity-for-expression' (or 'intrinsic-strength') score that encapsulates its gene-specific control state under the condition considered.
We denote the affinity of gene $i$ under growth condition $c$ by $w_i(c)$ (the notion of affinity for expression is not new, and was first suggested in  \cite{Maaloe1969}).
Our model assumes that the bio-synthetic resources of the cell (Ribosomes, RNA Polymerases, etc.) are distributed among the genes according to their affinities under the condition at hand.
The notion of affinities can thus reduce the number of parameters needed to predict expression levels markedly.
Instead of an expression level for every gene under each condition, there is only a need for the characterization of the affinities a gene may obtain under relevant environmental cues, a parameter set that is expected to be much smaller and easily characterized.
Figure \ref{fig:randpred} shows the prediction versus the actual concentration values of 9 random proteins in the data set from [ref Heinemann].\todo{Does our current data make this figure convincing?}


The resulting protein fraction, under a specific condition, is therefore its specific affinity under the condition, divided by the sum of all the affinities of all of the genes under that same condition.
Thus, if two proteins have the same affinity under some condition, they will occupy identical fractions out of the proteome under that condition.
If protein $A$ has twice the affinity of protein $B$ under a given condition, then the fraction $A$ occupies will be twice as large as the fraction occupied by $B$ under that condition, etc.


This relationship can be simply formulated as follows:
\begin{equation}
  \label{eq:concentration-ratio}
  p_i(c)=\frac{P_i(c)}{P(c)}=\frac{w_i(c)}{\sum_jw_j(c)}
\end{equation}
where $p_i(c)$ denotes the fraction of protein $i$ under condition $c$ out of the proteome, $P_i(c)$ denotes the mass of protein $i$ under condition $c$ per cell, $P(c)$ denotes the total mass of proteins per cell under condition $c$, and the sum, $\sum_jw_j(c)$, is taken over all the genes the cell has.


This equation implies that the observed fraction of a protein is determined by two factors, first, obviously, its own specific affinity that is present in the nominator, but second, and less intuitive and commonly thought of, the affinity of all of the other genes under the growth condition, as reflected by the denominator.


\subsubsection{A change in growth condition triggers changes in expression of specific proteins that indirectly affect all of the proteome}
Different environmental conditions may require the expression of different genes in order to achieve growth.
For example, comparing two growth media, one that includes amino-acids, and one that does not, it can be assumed that when amino-acids are present, no need exists for the cell to express amino-acids synthesizing enzymes, whereas when amino-acids are absent, these enzymes must be expressed.
Therefore, ideally, the cell will be able to sense the presence or absence of amino-acids in the growth media and, for the amino-acids synthesizing genes, down or up regulate their affinities accordingly.
If we now consider some arbitrary gene $i$, whose specific affinity is unaltered between these two conditions, we suggest that, other things being equal, its concentration will still change between the two conditions as the affinities of at least some of the other genes (the amino-acids synthesizing enzymes) change, changing the denominator in equation \ref{eq:concentration-ratio} and thus affecting the distribution of resources between all of the expressed genes.


Generalizing this notion, for every group of conditions, one could divide the proteins into those whose intrinsic affinity remains constant across all of the conditions, and to those whose intrinsic affinity changes (meaning their expression is actively regulated by the cell) between at least some of the conditions, as is shown in Figure \ref{fig:model}A.
An interesting consequence of the formulation in Equation \ref{eq:concentration-ratio} is that proteins whose intrinsic affinities remain constant across different growth conditions, also \todo{Figure 4 could include different shades of blue in panel A that will demonstrate this graphically} maintain their relative concentrations across these conditions with respect to each other.
Therefore, identifying a large group of proteins that maintain their relative concentrations across conditions (as was identified in section \ref{propchange}) may indicate that these proteins maintain their intrinsic affinities and that any changes in their absolute concentrations are in fact a passive outcome resulting from changes in the intrinsic affinities of other proteins.


\subsubsection{The observed growth rate is an outcome of proteome composition and environmental conditions}
While it is sometimes implied that different cellular components are regulated by the growth rate, here we consider the growth rate as an outcome of the environmental conditions that affect the proteome composition.
Specifically, we arrive at the doubling time as the result of dividing the total amount of proteins per cell by the amount of bio-synthesis machinery in that cell.
The larger the ratio of total proteins to bio-synthesis proteins is, the longer these bio-synthesis proteins will have to operate in order to duplicate the proteome, and thus the longer the doubling time of the cell will be.


To illustrate this assumption concretely, one could think about the total amount of proteins per cell (measured in amino-acids count) divided by the number of ribosomes in the cell.
Assuming each ribosome translates at an approximately condition-independent rate of about 20 amino-acids per second, and as the amount of actively translating ribosomes is also relatively condition-independent \cite{Philips2009} (\emph{ToDo:and its refs}), it follows that the doubling time is linearly dependent on the ratio of proteins to ribosomes in the biomass as illustrated in Figure \ref{fig:model}B.


Theoretically, the fastest doubling time a cell may have is the doubling time achieved when all of the proteome of the cell is the bio-synthetic machinery.
We denote this minimal doubling time by $T_B$.
If the bio-synthetic machinery is only half of the proteome, the doubling time will be $2T_B$ etc.


To integrate the notion of total protein to bio-synthetic protein ratio into our model, we make the following simplifying assumption:
There is a group of bio-synthetic genes (e.g. genes of the transcriptional and translational machineries) the affinities of which remain constant across different growth conditions, a.k.a. these genes are not actively differentially regulated across different conditions.
Furthermore, we assume that the machineries these genes are involved at operate at relatively constant rates and active to non-active ratios across conditions.
Under these assumptions we can define this group of bio-synthesis genes, $G_B$, such that, for every gene that belongs to this group, $k \in G_B$, its affinity, $w_k(c)$ is constant regardless of the condition, $c$.
\begin{equation}
  \label{eq:biosynth-def}
  w_k(c)=w_k
\end{equation}


To keep our notations short, we will define the (condition independent) sum over all of these bio-synthesis genes as the constant: $W_B = \sum_{k\in G_B}w_k$.


As these genes form the bio-synthesis machinery, and according to the assumptions presented above, it follows that the doubling time under a given condition, $\tau(c)$ will be proportional to the ratio of total protein to bio-synthesis protein under that condition, with the proportionality constant being $T_B$:
\begin{equation}
  \label{eq:gr-ratio}
  \tau(c) = T_B\frac{P(c)}{\sum_{k\in G_B}P_k(c)}=T_B\frac{\sum_jw_j(c)}{W_B}
\end{equation}
Therefore, the model implies that for conditions that require the expression of larger amounts of non-bio-synthetic genes (i.e. higher values in the sum over $w_j$ that are not in $W_B$), the resulting doubling time will be longer, i.e., the growth rate will be lower.


\subsubsection{The concentration of a non-differentially regulated protein is expected to increase with the growth rate} 
Recalling that the connection between the growth rate and the doubling time is: $g(c)=\frac{\ln(2)}{\tau(c)}$, we now combine Equation \ref{eq:concentration-ratio} with Equation \ref{eq:gr-ratio} to get that:
\begin{equation}
  \label{eq:default-response}
  p_i(c)=\frac{w_i(c)}{\sum_jw_j(c)}=\frac{w_i(c)}{W_B}\frac{W_B}{\sum_jw_j(c)}=\frac{w_i(c)}{W_B}\frac{T_B}{\ln(2)}g(c)
\end{equation}


Incorporating all the condition-independent constants ($W_B$, $T_B$, $\ln(2)$) into one term, $C$, we get that the predicted fraction of protein $i$ out of the proteome under condition $c$ is:
\begin{equation}
  \label{eq:final-conc}
  p_i(c)=Cw_i(c)g(c)
\end{equation}
which implies that, for every two conditions between which gene $i$ maintains its affinity, ($w_i(c_1)=w_i(c_2)$), the fraction protein $i$ occupies out of the proteome scales like the growth rate change between these two conditions.


To summarize, the simplistic model we have constructed predicts that, under no specific regulation, the fraction a protein occupies out of the proteome should scale with the growth rate.
A group of such proteins should therefore maintain their relative concentrations across conditions.
Finally, when the growth rate approaches zero, the fractions of such proteins, and thus their concentrations, should also approach zero.


However, as the analysis of experimental data in Figure \ref{fig:globalgrcorr} shows, while the concentration of many proteins does indeed scale linearly with growth rate, this scaling does not imply a drop to zero concentration at zero growth.
There are at least two factors that have been neglected in the model, but that can account for this result, and the analysis of their expected effects follows.


\subsubsection{Protein degradation differentiates between measured growth rate and biomass synthesis rate}
Accounting for protein degradation affects the expected concentration of non-differentially regulated proteins at zero growth rate.
Thus, accounting for protein degradation may serve as a partial explanation for the discrepancy between experimental data and predictions made by the model.


Assuming that protein degradation acts on all proteins in the same way, and that it is invariant in the growth condition, the effect of protein degradation can be understood as follows: at any time, some fraction of the entire proteome is degraded.
Therefore, the \emph{observed} growth rate, $g$, is, in fact, the amount of proteins produced \emph{minus} the amount of proteins degraded.
To illustrate, if the measured growth rate is zero, the implication is not that no proteins are produced, but rather that proteins are produced at exactly the same rate as they are degraded.


Integrating this notion into the model means that, where the equations previously referred to the observed growth rate, $g$, as the indicator of protein synthesis rate, they should in fact refer to the observed growth rate plus the degradation rate, as that is the real rate of protein synthesis.
Therefore, if we denote by $\alpha$ the degradation rate (assuming for now equal degradation rates for all genes and under all conditions), Equation \ref{eq:final-conc} should be rewritten as:
\begin{equation}
  \label{eq:final-conc-deg}
  p_i(c)=Cw_i(c)(\alpha+g(c))
\end{equation}
This equation yields better agreement with the experimental results as presented in Figure \ref{fig:globalgrcorr}, depending on the exact value set for degradation.
Degradation can thus explain why the concentration of non-differentially regulated proteins does not drop to zero when the growth rate is zero.
The actual values obtained for the data analyzed are $\alpha=0.34$ for \cite{Valgepea2013} and $\alpha=0.52$ for \cite{Heinemann2014}, corresponding to protein half lives times of $T_{\text{deg}}=2$ hours and $T_{\text{deg}}=1.3$ hours respectively.
As these values correspond to relatively short half lives times, protein degradation is probably only a partial explanation for the differences between the predictions of the model and the observations obtained from the experimental data.


\subsubsection{Slower biological processes rates at slower growth affect the relation between proteome composition and growth rate}
The simplistic model assumes that the doubling time is proportional to the ratio of total protein to bio-synthetic protein.
This assumption fails if the rate at which the bio-synthetic machinery operates changes across conditions.
Replacing this assumption by a dependence of bio-synthesis rate with growth rate (such that, the faster the growth, the faster the synthesis rates), will affect the resulting predictions as well.
Slower bio-synthesis rates under slower growth rates imply that, compared with the model prediction, higher fraction of bio-synthesis proteins is needed to achieve a given growth rate.
Thus, lower synthesis rates under slower growth rates will be reflected by a lower slope and higher interception point for non-regulated proteins than those predicted by the model, as is indeed the case in Figure \ref{fig:globalgrcorr}.
