The data we use includes the concentrations of proteins under different growth conditions, and the growth rate for every condition.
Given a threshold on the Pearson correlation with growth rate, one can focus on the group of proteins with a correlation with growth rate that is higher than the threshold.
For this group of proteins, a linear regression response can be calculated.
We define the explained variability by the growth rate, given a threshold, as the difference between the total variability of the group of proteins with a correlation higher than the threshold, and the variability remaining, when assuming these proteins scale with the growth rate according to the calculated linear response.
Dividing the explained variability by the total variability of the entire data set quantifies what fraction of the total variability in the data set is explained by considering linear scaling with growth rate for all the proteins with a correlation with growth rate higher than the threshold.
The optimal threshold is then defined as the threshold maximizing this fraction.
As different proteins have very different average concentrations, the aforementioned calculation may be biased towards proteins with higher average concentrations.
To avoid this effect, the analysis was performed on the normalized concentrations as defined in \ref{concacrossconds}.
Figure \ref{fig:threshold} shows the fraction of the variability explained as a function of the threshold used, as well as the fraction of the variability of the proteins included in the group for which the regression was calculated explained by the same regression line.


\subsection{Differences between the correlations found in the two data sets}
\label{heinemannchemo} 

The lower correlation and higher variability found in the data set from \cite{Heinemann2014} partially results from the variability in the conditions it contains as well as the higher number of conditions measured across a similar range of growth rates.
Specifically, as this data set includes measurements under different carbon sources, as opposed to the data set from \cite{Valgepea2013}, that uses the same carbon source on all measurements, a larger variability in expression patterns is expected.
Restricting the analysis of the data set from \cite{Heinemann2014} only to chemostat conditions supports this suggestion and shows much less variability as is shown in Figure \ref{fig:growthcorrchemo}.
