The theoretical lower limit of the doubling time, $T_B$, will be achieved when all of the proteome of the cell is the bio-synthetic machinery.
If the bio-synthetic machinery is only half of the proteome, the doubling time will be $2T_B$ etc.

To integrate the notion of total protein to bio-synthetic protein ratio into our model, we make the following simplifying assumption:
There is a group of bio-synthetic genes (e.g. genes of the transcriptional and translational machineries) the affinities of which remain constant across different growth conditions, that is, these genes are not actively differentially regulated across different conditions.
Furthermore, we assume that the machineries these genes are involved in, operate at relatively constant rates and active to non-active ratios across conditions, as has been shown for ribosomes \cite{Bremer1987}.