\subsubsection{A large fraction of the proteome is positively correlated with growth rate}
In each data set, the growth rate and proteome composition were measured for several conditions.
We calculated the Pearson correlation of every protein with the growth rate, conducting the analysis seperately for each data set.
A histogram of the distribution of the correlations is shown in Figure \ref{fig:growthcorr}.
We find that $\approx 1/2$ to $\approx 1/3$ of the proteins ($881$ out of $1654$ measured in the data set from \cite{Heinemann2014}, hereafter referred to as H, and $296$ out of $919$ in the data set from \cite{Valgepea2013}, hereafter referred to as V) are strongly positively correlated with the growth rate.
A strong correlation with growth rate was defined as a correlation of $R\geq 0.8$ for the V data set, and $R\geq 0.25$ for the H data set.
The thresholds for defining strong correlation with growth rate were chosen to be the values maximizing the explained variability by a simple linear regression for these proteins, out of the total variability of the proteome (See \ref{corrthreshold} for exact definition and calculation used).
For further comparison and analysis of the causes underlying the differences between the two data sets as reflected in Figure \ref{fig:growthcorr} see section \ref{heinemannchemo}.
Notably, in both data sets, the proteins that have a high correlation with the growth rate are involved in different cellular functions and span different functional groups (See tables \ref{tab:corrbreakdownh} and \ref{tab:corrbreakdownv}).
Previous studies already found that ribosomal proteins are strongly positively correlated with growth rate \cite{Pedersen1978a, ingraham1983growth, Klumpp2008}.
However, we find that the strongly positively correlated with growth rate group of proteins includes more proteins than the 56 ribosomal proteins, the vast majority of which we also find to be strongly correlated with the growth rate.
Specifically, the proteins that we find to be strongly positively correlated with growth rate are not generally expected to be co-regulated.