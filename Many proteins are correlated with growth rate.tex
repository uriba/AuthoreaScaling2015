\subsubsection{A large fraction of the proteome is positively correlated with growth rate}
Our model predicts that a large fraction of the proteins should increase in concentration with the growth rate.
To test this prediction, we calculated the Pearson correlation of every protein with the growth rate, conducting the analysis separately for each data set.
A histogram of the distribution of the correlations is shown in Figure \ref{fig:growthcorr}.
We find that more than a third of the proteins (\hGlobal{} out of \hTotal{} measured in the data set from \cite{Heinemann2015}, hereafter referred to as H, and \vGlobal{} out of \vTotal{} in the data set from \cite{Valgepea2013}, hereafter referred to as V) have a strong positive ($>0.5$) correlation with the growth rate.
Further discussion of the choice of threshold for defining strong correlation with the growth rate is in section \ref{corrthreshold}.
We note that these amounts are much higher than those obtained for randomized data sets (\hGlobalShuff{} and \vGlobalShuff{} strongly positively correlated proteins for the two data sets, respectively, as is further discussed in section \ref{randanalysis}).
Further comparison and analysis of the causes underlying the differences between the two data sets as reflected in Figure \ref{fig:growthcorr} are in section \ref{heinemannchemo}.
Notably, in both data sets, the proteins that have a high correlation with the growth rate are involved in many and varied cellular functions and span different functional groups (See tables \ref{tab:corrbreakdownh} and \ref{tab:corrbreakdownv}).

Previous studies already found that ribosomal proteins are strongly positively correlated with growth rate \cite{Pedersen1978a, ingraham1983growth, Klumpp2008}.
Our analysis agrees with these findings as we find the concentration of the vast majority of the ribosomal proteins to be strongly positively correlated with growth rate (\hCorrRibs{} out of \hRibs{} in the data set from \cite{Heinemann2015} and \vCorrRibs{} out of \vRibs{} in the data set from \cite{Valgepea2013}).
However, we also find that the group of proteins strongly positively correlated with growth rate includes many more proteins than the ribosomal proteins, as can be seen in tables \ref{tab:corrbreakdownh} and \ref{tab:corrbreakdownv}).
Importantly, the proteins that we find to be strongly positively correlated with growth rate are not generally expected to be co-regulated, and their behavior does not seem to be the result of any known transcription factor or regulation cluster response \cite{23203884}.