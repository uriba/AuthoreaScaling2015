\subsubsection{A large fraction of the proteome is positively correlated with growth rate}
Our model predicts that a large portion of the proteins should increase in fraction with the growth rate, as that is the expected change for proteins that are not specifically regulated between conditions.
To test this prediction, we calculated the Pearson correlation of every protein with the growth rate (Figure \ref{fig:growthcorr}).
We find that about a third of the proteins (\hGlobal{} out of \hTotal{} measured in the data set from \cite{Heinemann2015}, and \vGlobal{} out of \vTotal{} in the data set from \cite{Peebo_2015}) have a strong positive ($>0.5$, see also \ref{corrthreshold}) correlation with the growth rate.
We note that these amounts are much higher than those obtained for randomized data sets (\hGlobalShuff{} and \vGlobalShuff{} strongly positively correlated proteins for the two data sets, respectively, as is further discussed in section \ref{randanalysis}).
Strong negative correlation with growth rate is much less common in the data set from \cite{Heinemann2015}.
It is common in the data set from \cite{Peebo_2015}, where we speculate that it results from the specific way by which growth rate was controlled, namely by implicitly controlling nutrient concentration via an accelero-stat.
The control of growth rate by gradual changes to nutrient concentration may naturally lead to gradual changes in protein levels, both increasing and decreasing, an effect that is expected under any regulatory scheme and is thus irrelevant for our analysis. 
Further comparison and analysis of the causes underlying the differences between the two data sets as reflected in Figure \ref{fig:growthcorr} are in section \ref{heinemannchemo}.
Notably, in both data sets, the proteins that have a high correlation with the growth rate are involved in many and varied cellular functions and span different functional groups (See tables \ref{tab:corrbreakdownh} and \ref{tab:corrbreakdownv}).

Previous studies already found that ribosomal proteins are strongly positively correlated with growth rate \cite{Pedersen1978a, ingraham1983growth, Klumpp2008}.
Our analysis agrees with these findings as we find the fraction of the vast majority of the ribosomal proteins to be strongly positively correlated with growth rate (\hCorrRibs{} out of \hRibs{} in the data set from \cite{Heinemann2015} and \vCorrRibs{} out of \vRibs{} in the data set from \cite{Peebo_2015}).
However, we also find that the group of proteins strongly positively correlated with growth rate includes many more proteins than the ribosomal proteins (tables \ref{tab:corrbreakdownh} and \ref{tab:corrbreakdownv}).
Importantly, the proteins that we find to be strongly positively correlated with growth rate are not generally expected to be co-regulated, and their behavior does not seem to be the result of any known transcription factor or regulation cluster response \cite{23203884}.