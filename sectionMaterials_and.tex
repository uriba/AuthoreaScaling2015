\section{Materials and Methods}
Detail data sources, software and algorithms used.
\subsubsection{Normalizing protein concentration across conditions}
\label{concacrossconds} 

When comparing the response of different proteins to the growth rate, a normalization is required to compensate for differences in their mean concentrations.
For example, consider two proteins, $A$ and $B$ measured under two conditions, $C_1$ and $C_2$.
Assume that the measured fractions out of the proteome of these two proteins under the two conditions were $0.001$ and $0.002$ for $A$ under $C_1$ and $C_2$ respectively, and $0.01$ and $0.02$ for $B$ under $C_1$ and $C_2$ respectively.
These two proteins therefore respond in the same way across the two conditions, namely, they double their fraction in the proteome in $C_2$ compared with $C_1$.
The normalization procedure scales the data so as to reveal this identity in response.
Dividing the fraction of each protein out of the proteome by the average fraction of that protein across conditions, we get normalized fractions that are $\frac{2}{3}$ for both $A$ and $B$ at $C_1$ and $\frac{4}{3}$ for both $A$ and $B$ at $C_2$ showing their identical response.


This normalization procedure has been used prior to calculating the slopes of the regression lines best describing the change in fraction out of the proteome of every protein as a function of the growth rate.
Furthermore, when analyzing the variability explained by linear regression on the sum of concentrations of all proteins presenting a high correlation with the growth rate, the same normalization procedure was made in order to avoid biases resulting from the high abundance of a few proteins in that group.
\subsubsection{Filtering out conditions from the Heinemann data set}
\label{heinemanncond} 

The \cite{Heinemann2014} data set contains proteomic data measurements under 19 different environmental conditions.
However, some of these conditions are expected to affect the proteome composition in ways not related to the actual growth rate.
Specifically, our model assumes that bio-synthesis rates are condition independent and, at most, change only with growth rate.
That will likely not be the case under specific kinds of conditions such as osmotic pressure, low pH and heat stress.
Additionally, our model assumes exponential growth, implying that measurements taken at stationary phase are expected to differ from simple extrapolation of the model to zero growth rate.
We have therefore omitted such conditions from our analysis and analyzed only those conditions under which our assumptions for bio-synthesis rates being either constant or have a simple relation to the growth rate do hold \todo{need to add these analyses to the SI}.


As, out of the conditions measured in the \cite{Heinemann2014} data set, growth in LB media presented a much faster growth rate than the rest of the conditions measured, it dominated the behavior and trends calculated, when included in the analysis.
While including the data on growth in LB does not qualitatively changes the observed results, such analysis is less statistically robust.
We have therefore omitted LB growth data in the main analysis yet present it in the SI.
Including LB growth results in a much smaller set of proteins with a strong positive correlation with growth (as many of the proteins in that group in the slower conditions get down-regulated in LB, significantly reducing their Pearson correlation with growth rate).
On the other hand, the proteins that remain strongly positively correlated with growth rate when LB is included in the analysis show a higher correlation compared with the strongly correlated with growth rate proteins under the slower conditions.
Furthermore, despite the decrease in the number of proteins that are strongly positively correlated with growth when LB is included in the analysis, these proteins form up to $50\%$ of the proteome under LB.


\subsubsection{Calculation of protein concentration}
\label{protconc} 

In this study, we use the mass ratio of a specific protein to the mass of the entire proteome, per cell, as our basic measure for the bio-synthetic resources a specific protein consumes out of the bio-synthetic capacity of the cell.
We find this measure to be the best representation of the meaning of a fraction a protein occupies out of the proteome.
However, we note that if initiation rates are limiting, and not elongation rates, then using molecule counts ratios (the number of molecules of a specific protein divided by the total number of protein molecules in a cell) rather than mass ratios may be a better metric.
We compared these two metrics and, while they present some differences in the analysis, they do not qualitatively alter the observed results \todo{add to SI a comparison}.


There are different, alternative ways to assess the resources consumed by a specific protein out of the resources available in the cell.
On top of the measures listed above, one could consider either the total mass or molecule count of a specific protein out of the biomass, or out of the dry weight of the cell, both of which take into account the ratio of total protein to biomass or dry weight which was neglected in our analysis.
Moreover, one can consider specific protein mass or molecule count per cell, thus reflecting changes in cell size across conditions.
Our analysis focused on the relations between different proteins and resource distribution inside the proteome, and thus avoided such metrics.