Many studies monitored the ribosome concentration in cells and its interdependence with growth rate \cite{Schaechter1958,Bremer1987,Zaslaver2009,Scott2010,Valgepea2013,Peebo_2015,Hui_2015}(many of them indirectly).
While in all of these studies a linear dependence of ribosome concentration with growth rate was observed, in some cases different slopes and interception points were found to describe this linear dependence, compared with the observations in our study.
A discussion of various reasons that may underlie these differences is given in section \ref{ribosomeconc}.

Interestingly, our model suggests that a linear correlation between ribosomal proteins and the growth rate might be achieved without special control mechanisms.
Nonetheless, many such mechanisms have been shown to exist \cite{Nomura1984,25149558}.
We stress that the existence of such mechanisms does not contradict the model.
Mechanisms for ribosomal proteins expression control may still be needed to achieve faster response under changing environmental conditions or a tighter regulation to avoid unnecessary production and reduce translational noise.
Furthermore, such mechanisms may be crucial for synchronizing the amount of rRNA with ribosomal proteins as the two go through different bio-synthesis pathways.
Nevertheless, the fact that many non-ribosomal proteins share the same response as ribosomal proteins do, poses interesting questions regarding the scope of such control mechanisms, their necessity and the trade-offs involved in their deployment.