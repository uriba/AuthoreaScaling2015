\subsubsection{The concentration of a non-differentially regulated protein is expected to increase with the growth rate} 
Recalling that the connection between the growth rate and the doubling time is: $g(c)=\frac{\ln(2)}{\tau(c)}$, we now combine Equation \ref{eq:concentration-ratio} with Equation \ref{eq:gr-ratio} to get that:
\begin{equation}
  \label{eq:default-response}
  p_i(c)=\frac{w_i(c)}{\sum_jw_j(c)}=\frac{w_i(c)}{W_B}\frac{W_B}{\sum_jw_j(c)}=\frac{w_i(c)}{W_B}\frac{T_B}{\ln(2)}g(c)
\end{equation}


Incorporating all the condition-independent constants ($W_B$, $T_B$, $\ln(2)$) into one term, $C$, we get that the predicted fraction of protein $i$ out of the proteome under condition $c$ is:
\begin{equation}
  \label{eq:final-conc}
  p_i(c)=Cw_i(c)g(c)
\end{equation}
which implies that, for every two conditions between which gene $i$ maintains its affinity, ($w_i(c_1)=w_i(c_2)$), the fraction protein $i$ occupies out of the proteome scales like the growth rate change between these two conditions.


To summarize, the simplistic model we have constructed predicts that, under no specific regulation, the fraction a protein occupies out of the proteome should scale with the growth rate.
A group of such proteins should therefore maintain their relative concentrations across conditions.
Finally, when the growth rate approaches zero, the fractions of such proteins, and thus their concentrations, should also approach zero.


However, as our analysis of experimental data in Figure \ref{fig:globalgrcorr} shows, while the concentration of many proteins does indeed scale linearly with growth rate, this scaling does not imply a drop to zero concentration at zero growth.
There are at least two factors that have been neglected in the model, but that can account for this result, and the analysis of their expected effects follows.