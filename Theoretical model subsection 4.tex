\subsubsection{The fraction of a non-differentially regulated protein is expected to increase with the growth rate} 
Recalling that the connection between the growth rate and the doubling time is: $g(c)=\frac{\ln(2)}{\tau(c)}$, we now combine Equation \ref{eq:concentration-ratio} with Equation \ref{eq:gr-ratio} to get a prediction for the single protein fractions $p_i$:
\begin{equation}
  \label{eq:default-response}
  p_i(c)=\frac{w_i(c)}{\sum_jw_j(c)}=\frac{w_i(c)}{W_B}\frac{W_B}{\sum_jw_j(c)}=\frac{w_i(c)}{W_B}\frac{T_B}{\ln(2)}g(c)
\end{equation}

By incorporating all the condition-independent constants ($W_B$, $T_B$, $\ln(2)$) into one term, $A$, we can simplify to:
\begin{equation}
  \label{eq:final-conc}
  p_i(c)=Aw_i(c)g(c)
\end{equation}
Hence, for every two conditions between which gene $i$ maintains its affinity, ($w_i(c_1)=w_i(c_2)$), the fraction $p_i(c)$ protein $i$ occupies in the proteome scales in the same way as the growth rate between these two conditions.

To summarize, the simplified model we have constructed predicts that, under no specific regulation, the fraction a non-regulated protein occupies out of the proteome should scale with the growth rate.
A group of such proteins should therefore maintain their relative ratios across conditions.