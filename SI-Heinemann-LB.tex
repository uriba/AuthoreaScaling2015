\subsection{Analysis including LB condition}
\label{lbanalysis}
Due to the fast growth rate under LB, compared with the other conditions measured in the data set from \cite{Heinemann2015} it was not included in our primary analysis as was noted in section \ref{heinemanncond}.
Including LB growth results in a much smaller set of proteins with a strong positive correlation with growth, as many of the proteins in that group in the slower conditions get down-regulated in LB, significantly reducing their Pearson correlation with growth rate.
For example, the Pearson correlation with growth rate of gapA, involved in glycolisys, drops from 0.73 to 0.35 when LB is included.
Another such example is glyA, involved in serine and threonine metabolism, that has a correlation with
growth rate of -0.12 when LB is included in the analysis vs. a correlation of 0.7 without it.

Figure \ref{fig:LB} shows the implications of including LB in the analysis.
As can be seen, many proteins are now less correlated with growth rate due to down regulation under LB.
However, despite having fewer proteins being strongly positively correlated with growth (\hGlobalLB{} vs. \hGlobal{}) and despite the accumulated fraction of these proteins being lower under the slower growth conditions ($\approx20\%$ vs. $\approx25\%$), these proteins do occupy $>50\%$ out of the proteome under fast growth in LB.