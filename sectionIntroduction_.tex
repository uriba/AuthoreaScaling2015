\section{Introduction}
A fundamental system-level challenge for cell physiology is the achievement of proper function in the face of various environmental conditions.
It has been established for many years that in different environments cells differ in many properties, including their shape, size, growth rate, and macro-molecular composition \cite{Schaechter1958,Maaloe1969,Churchward1982,Pedersen1978a,ingraham1983growth,Bremer1987}, with strong interdependence between these parameters.


Early on it was found that the expression of some genes is coordinated with growth rate, rather than with the specific environment.
Classical experiments in bacteria, by researchers from what became known as the Copenhagen school, have shown that ribosome concentration (inferred from the RNA to protein ratio in cells) increases in proportion to growth rate\cite{Schaechter1958}.
The search for mechanisms in \emph{E.coli} that underlie this observation yielded several candidates.
Specifically, coordination between ribosome production and growth rate was attributed both to the pools of purine nucleotides \cite{Gourse1996,Gaal1997}, and the tRNA pools through the stringent response \cite{Chatterji2001,Brauer2008a}.
For a more thorough review see \cite{Nomura1984}.
The logic behind this observed increase in concentration is that, given that translation rates and active ribosomes fraction remain relatively constant across conditions, a larger fraction of ribosomes out of the proteome is needed in order to achieve faster growth \cite{Neidhardt1999,Dennis2004,Zaslaver2009}.


In the last two decades, with the development of the ability to measure genome-wide expression levels, it was found that coordination of gene expression (measured through mRNA levels and promoter-reporter libraries) and growth rate is not limited to ribosomes and ribosomal genes.
In \emph{E.coli}, the expression of catabolic and anabolic genes is coordinated with growth rate, and suggested to be mediated by cAMP \cite{Saldanha2004}.
In \emph{S.cerevisiae}, it was shown that a surprisingly large fraction of the genome changes its expression levels in response to environmental conditions in a manner strongly correlated with growth rate \cite{Keren2013a,Gasch2000,Castrillo2007,Zaslaver2009,Gerosa2013}.
Studies examining the interplay between global and specific modes of regulation, suggested that global factors play a major role in determining the expression levels of genes \cite{Gasch2000,Klumpp2009a,Scott2010,Berthoumieux2013}.
In \emph{E.coli}, this was mechanistically attributed to changes in the pool of RNA polymerase core and sigma factors \cite{Klumpp2008}.
In \emph{S.cerevisiae}, it was suggested that differences in histone modifications around the replication origins \cite{Regenberg2006} or translation rates \cite{Gasch2000} across conditions may underlie the same phenomenon.
Important advancements in \emph{E.coli} were achieved by analyzing measurements of fluorescent reporters through a simplified model of gene expression built upon the empirical scaling with growth rate of different cell parameters (such as gene dosage, transcription rate and cell size)\cite{Klumpp2009a}.
These studies suggest that the expression of all genes changes with growth rate, with different architectures of regulatory networks yielding differences in the direction and magnitude of these changes. 


Despite these advancements, many gaps remain in our understanding of the connection between gene expression and growth rate.
Primarily, it is unclear what is the degree of interconnection between gene expression and growth rate.
Is it unique to specific groups of genes or is it a more global phenomenon shared across most genes in the genome?
How much of the variability observed in gene expression patterns across different growth conditions results from active adaptation to the specific condition, and how much results from global, condition-independent, response.
Genome-wide proteomic data sets, such as those generated by mass-spectrometry, which probe the proteome composition at different growth rates, offer potential insights into these questions.

In this work we present a parsimonious model, which does not require fine-tuning according to organism and condition-specific parameters, that quantitatively predicts the relationship between gene regulation, protein abundance and growth rate.
Our model provides a baseline for the behavior of endogenous genes in conditions between which they are not differentially regulated, on top of which different regulatory aspects can be added.
It suggests that positive correlation of protein concentration with growth rate is a system-emerging property that is the result of passive redistribution of resources, without need for specific regulation mechanisms.
In order to validate the model, we analyzed two recently published proteomic data sets of \emph{E.coli} under different growth conditions \cite{Valgepea2013, Heinemann2014}.
We find a statistically significant, coordinated, positive correlation between growth rate and the protein concentration of many genes, from diverse functional groups.
However, this response accounts for a relatively small fraction of the total variability of the proteome across the different growth conditions for which these data sets were obtained.
Our analysis suggests that experimental noise may underly this relatively poor explenatory power, concluding that more data will be required in order to support or refute the model we present.
