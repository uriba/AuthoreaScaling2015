Most proteins show changes in level across growth conditions.
Many of these changes seem to be coordinated with the specific growth rate rather than the growth environment or the protein function.
Although cellular growth rates, gene expression levels and gene regulation have been at the center of biological research for decades, there are only a few models using the value of the specific growth rate to partially predict the proteome composition.

We present a simple model that predicts a widely coordinated increase in the fraction of many proteins proportionally with the growth rate.
The model reveals how passive redistribution of resources, due to active regulation of only a few proteins, can have quantitatively predictable proteome wide effects.
Our model provides a potential explanation for why and how such a coordinated response of a large fraction of the proteome to the specific growth rate arises under different environmental conditions.
The simplicity of our model can also be useful by serving as a baseline null hypothesis in the search for active regulation.
We exemplify the usage of the model by analyzing the relationship between growth rate and proteome composition for the model microorganism \emph{E.coli} as reflected in two recent proteomics data sets spanning various growth conditions.
We find that the fraction out of the proteome of a large number of proteins, and from different cellular processes, increases proportionally with the growth rate.
Notably, ribosomal proteins, which have been previously repetitively reported to increase in fraction with growth rate, are only a small part of this group of proteins.
We suggest that, although the fractions of many proteins change with the growth rate, such changes could be part of a global effect, not requiring specific cellular control mechanisms.