Most proteins show changes in level across growth conditions.
Many of these changes seem to be coordinated with the growth rate rather than the specific environment or the protein function.
Although cellular growth rates, gene expression levels and gene regulation have been at the center of biological research for decades, there are only a few models using the value of the growth rate to partially predict protein levels.

We present a simple model that predicts a widely coordinated increase in the concentration of many proteins proportionally with the growth rate. The model reveals how passive redistribution of resources, due to active regulation of only a few proteins, can have quantitatively predictable proteome wide effects.
Our model provides a potential explanation for why and how such a coordinated response of a large fraction of the proteome to the growth rate arises under different environmental conditions.
The simplicity of our model can also be useful by serving as a baseline null hypothesis in the search for active regulation.
We exemplify the usage of the model by analyzing the relationship between growth rate and proteome composition for the model microorganism \emph{E.coli} as reflected in two recent proteomics data sets spanning various growth conditions.
We find that the cellular concentration of a large fraction of the proteins, and from different cellular processes, increases proportionally with the growth rate. Notably, ribosomal proteins are only a small fraction of this group of proteins.
Despite the large fraction of proteins that display this coordinated response, this response only accounts for a relatively small fraction of the overall variability in the proteome across different growth conditions, possibly due to experimental noise.
We suggest that, although the concentrations of many proteins change with the growth rate, such changes could be part of a global effect, not requiring specific cellular control mechanisms.