In order to compare the responses of different proteins across conditions, we divided the concentration of every protein, under every condition by its average concentration across all of the conditions (see \ref{concacrossconds} for further details).
The resulting normalized concentrations across conditions represent the concentrations of the specific protein under every condition, relative to its mean concentration across all conditions.
Under this metric, sharing similar responses among a group of proteins implies that proteins in the group are coordinated.
The relative ratios between such coordinated proteins are determined by the average concentration of each of them across the different environmental conditions.
For the group of proteins that were found to be strongly positively correlated with growth rate, we calculated the slope of a linear regression line for the normalized concentration vs. the growth rate for every one of these proteins and plotted the result in Figure \ref{fig:globalfit}.
The resulting distributions reveal that the slopes of $\approx \frac{2}{3}$ of the protein lie in the range $(0.5,2)$ and that the highest slopes are $\approx 5$.