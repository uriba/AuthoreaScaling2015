In order to examine how similar the behavior with growth rate is for the group of proteins that we found to be strongly positively correlated with growth rate, we divided the fraction of every such protein, in every condition, by its average fraction across all of the conditions (see \ref{concacrossconds} for further details).
The resulting normalized fractions across conditions represent the fractions of the specific protein under every condition, relative to its mean fraction across all conditions facilitating comparisons between proteins with differing expression levels.
Next, we calculated the slope of a linear regression line for the normalized fraction vs. the growth rate for every one of these proteins and plotted the result in Figure \ref{fig:globalfit}.
The resulting distributions reveal that the slopes of $\approx \frac{2}{3}$ of the proteins lie in the range $(0.5,2)$ and that the highest slopes are $\approx 5$.
A slope of $0.5$ means that the fraction of the protein changes by $\pm12\%$ around its average fraction in the range of growth rates measured, whereas a slope of $2$ indicates a change of $\pm50\%$ in the same range of growth rates.
This implies that the relative amounts of proteins with slopes in the range of $(0.5,2)$ change by at most just over 2-fold over the range of growth rates measured.