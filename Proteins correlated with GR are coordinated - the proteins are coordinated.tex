In order to examine how similar the behavior with growth rate is for the group of strongly positively correlated proteins, we normalized each of them to its mean abundance (\ref{concacrossconds}) and calculated the slope of a linear regression line for the normalized fraction vs. the growth rate (Figure \ref{fig:globalfit}).
The slopes of $\approx \frac{2}{3}$ of the proteins lie in the range $(0.5,2)$ with the highest slopes being $\approx 5$.
A slope of $0.5$ means that the fraction of the protein changes by $\pm12\%$ around its average fraction in the range of growth rates measured, whereas a slope of $2$ indicates a change of $\pm50\%$.
Hence, the relative amounts of proteins with slopes in the range of $(0.5,2)$ change by at most just over 2-fold over the range of growth rates measured.