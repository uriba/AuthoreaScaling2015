Next we examined how the response of the strongly correlated proteins relates to the well-studied response of ribosomes concentration.
To that end, we performed the same analysis of slopes, restricting it to ribosomal proteins alone, as is shown by the stacked green bars in Figure \ref{fig:globalfit}.
We find that, on average, strongly correlated proteins scale in the same way as ribosomal proteins do (mean slope of $1$ with standard deviation of $0.56$ for ribosomal proteins vs. $1.6$ and $0.87$ in the data from \ref{Heinemann2015}, and mean slope of $1.51$ with standard deviation of $0.22$ for ribosomal proteins vs. $1.34$ and $0.79$ in the data from \ref{Valgepea2013}. See also Figure \ref{fig:ribsnonribs}), implying that the observed response of ribosomal proteins to growth rate is not unique and is coordinated with a much larger fraction of the proteome, thus encompassing many more cellular components.