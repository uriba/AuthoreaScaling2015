In this work we present a parsimonious model, which does not require condition-specific parameters, that quantitatively predicts the relationship between protein abundance and growth rate in the absence of gene-specific changes in regulation.
Our model provides a baseline for the behavior of endogenous genes in conditions between which they are not differentially regulated, on top of which different regulatory aspects can be added.
The model predicts an increase in protein concentration with growth rate as an emerging property that is the result of passive redistribution of resources, without need for specific regulation mechanisms.
In order to exemplify and expore the scope of validity of the model, we analyzed two recently published proteomic data sets of \emph{E.coli} under different growth conditions \cite{Valgepea2013,Heinemann2015}.
We find a statistically significant, coordinated, positive correlation between growth rate and the protein concentration of many genes, from diverse functional groups.
However, this response accounts for a relatively small fraction of the total variability of the proteome across the different growth conditions for which these data sets were obtained.
Our analysis suggests that experimental noise may underly this relatively poor explanatory power, concluding that more data will be required in order to support or refute the model we present.
