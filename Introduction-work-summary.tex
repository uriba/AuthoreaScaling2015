In this work we present a parsimonious model, which does not require condition-specific parameters, that quantitatively predicts the relationship between protein abundance and growth rate in the absence of gene-specific changes in regulation.
Our model provides a baseline for the behavior of endogenous genes in conditions between which they are not differentially regulated, on top of which different regulatory aspects can be added.
The model predicts an increase in protein concentration with growth rate as an emerging property that is the result of passive redistribution of resources, without need for specific regulation mechanisms.
In order to exemplify and expore the scope of validity of the model, we analyzed two recently published proteomic data sets of \emph{E.coli} under different growth conditions \cite{Valgepea2013,Heinemann2015}.
We find a statistically significant, coordinated, positive correlation between growth rate and the protein concentration of many genes, from diverse functional groups.
Although this response accounts for a relatively small fraction of the total variability of the proteome it is highly significant, accounts for over $50\%$ of the proteome and includes the well-studied ribosomal proteins.
Our analysis suggests that, while changes in the proteome composition may seem complex, for a large number of proteins and under many conditions, they can be attributed to a linear, coordinated, increase with growth rate, at the expense of other, down-regulated proteins.
The well studied scaling of ribosomes concentration with growth rate can be considered one manifestation of this more general phenomena.