In this work we present a parsimonious model that quantitatively predicts the relationship between protein abundance and specific growth rate in the absence of gene-specific changes in regulation.
Our model provides a baseline for the behavior of endogenous genes in conditions between which they are not differentially regulated, without the need for condition-specific parameters.
The model predicts an increase in protein fraction with specific growth rate as an emerging property that is the result of passive redistribution of resources, without need for specific regulation mechanisms.
On top of this baseline model, different regulatory aspects can be added.
We tested the model against two recently published proteomic data sets of \emph{E.coli} spanning different growth conditions\cite{Peebo_2015,Heinemann2015}.
We find a coordinated, positive correlation between the specific growth rate and the fraction of many proteins, from diverse functional groups, out of the proteome.
Although this response accounts for a relatively small part of the total variability of the proteome it is highly significant, as it describes the behavior of over $50\%$ of the proteome including the well-studied ribosomal proteins.
Our analysis suggests that, even if changes in the proteome composition may seem complex, for a large number of proteins and under many conditions they can be attributed to a linear, coordinated, increase with growth rate, at the expense of other, down-regulated proteins.
The well studied scaling of ribosome concentration with growth rate can be considered one manifestation of this more general phenomena.