To give some intuition about the meaning of the values of slopes in figure \ref{fig:globalfit}, consider the following comparison between proteins with normalized slopes of $0.5$ and $2$, values that represent the range at which most strongly positively correlated with growth rate proteins lie.
A protein with a normalized slope of $0.5$ will change in concentration from $\frac{7}{8}$ of its mean concentration at the slowest growth rate measured ($\mu \approx 0.1$), to $\frac{9}{8}$ of its mean concentration at the fastest growth rate ($\mu \approx 0.6$).
A protein with a normalized slope of $2$ will have concentrations in the range $\frac{1}{2}$ to $\frac{3}{2}$ of its mean concentration across the same range of growth rates.
Such changes are relatively small compared with the known levels of noise in MS whole proteome measurements.
The ratio between proteins with such slopes of $0.5$ and $2$ lies in the relatively narrow range of $\frac{3}{4}$ to $\frac{7}{4}$ of the ratio between their mean concentrations, implying their relative amounts will change by at most just over 2-fold over the range of growth rates measured.