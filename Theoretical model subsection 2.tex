\subsubsection{A change in growth condition triggers changes in expression of specific proteins that indirectly affect the whole proteome}
Different environmental conditions require the expression of different genes.
For example, the expression of amino-acids synthesizing enzymes is required only in culture media lacking amino-acids \cite{24656150,10515934}.
Therefore, the cell can infer the presence or absence of amino-acids in the growth media and, regulate the affinities of the synthesizing genes accordingly.
If we consider a gene $i$, whose specific affinity is not dependent on the presence of amino-acids, we suggest that its fraction will still change between the two conditions as the affinities of other condition specific genes change, thereby redirecting the bio-synthetic capacity.
In mathematical terms this will chang the denominator in equation \ref{eq:concentration-ratio} and thus affect the distribution of resources between all of the expressed genes.


Generalizing this notion, we can divide the proteins into those whose intrinsic affinity remains constant across all of the considered conditions, and those whose intrinsic affinity changes between at least some of the conditions (Figure \ref{fig:model}).
An interesting consequence is that proteins whose intrinsic affinities remain constant also maintain their relative ratios across these conditions with respect to each other, as observed experimentally in \emph{S.cerevisae} in \cite{Keren2013}.
  