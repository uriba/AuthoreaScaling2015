\subsubsection{A change in growth condition triggers changes in expression of specific proteins that indirectly affect all of the proteome}
Different environmental conditions require the expression of different genes in order to achieve growth.
For example, comparing two growth media, one that includes amino-acids, and one that does not, it can be assumed that when amino-acids are present, no need exists for the cell to express amino-acids synthesizing enzymes, whereas when amino-acids are absent, these enzymes must be expressed.
Therefore, ideally, the cell will be able to sense the presence or absence of amino-acids in the growth media and, for the amino-acids synthesizing genes, down or up regulate their affinities accordingly.
If we now consider some unrelated gene $i$, whose specific affinity is unaltered between these two conditions, we suggest that its concentration will still change between the two conditions as the affinities of at least some of the other genes (the amino-acids synthesizing enzymes) change, changing the denominator in equation \ref{eq:concentration-ratio} and thus affecting the distribution of resources between all of the expressed genes.


Generalizing this notion, for every group of conditions, one could divide the proteins into those whose intrinsic affinity remains constant across all of the conditions, and to those whose intrinsic affinity changes (meaning their expression is actively regulated by the cell) between at least some of the conditions, as is shown in Figure \ref{fig:model}A.
An interesting consequence of the formulation in Equation \ref{eq:concentration-ratio} is that proteins whose intrinsic affinities remain constant across different growth conditions, also maintain their relative concentrations across these conditions with respect to each other.